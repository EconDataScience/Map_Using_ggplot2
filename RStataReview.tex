\documentclass[12pt]{article}%
\usepackage{amsfonts}
\usepackage{fancyhdr}
\usepackage{comment}
\usepackage{listings}
\usepackage[T1]{fontenc}
\usepackage{graphicx}
\usepackage[utf8]{inputenc}
\usepackage[a4paper, top=2.5cm, bottom=2.5cm, left=2.2cm, right=2.2cm]%
{geometry}
\usepackage{times}
\usepackage{pdfpages}
\usepackage{amsmath}
\usepackage{changepage}
\usepackage{amssymb}
\usepackage{graphicx}%
\setcounter{MaxMatrixCols}{30}
\newtheorem{theorem}{Theorem}
\newtheorem{acknowledgement}[theorem]{Acknowledgement}
\newtheorem{algorithm}[theorem]{Algorithm}
\newtheorem{axiom}{Axiom}
\newtheorem{case}[theorem]{Case}
\newtheorem{claim}[theorem]{Claim}
\newtheorem{conclusion}[theorem]{Conclusion}
\newtheorem{condition}[theorem]{Condition}
\newtheorem{conjecture}[theorem]{Conjecture}
\newtheorem{corollary}[theorem]{Corollary}
\newtheorem{criterion}[theorem]{Criterion}
\newtheorem{definition}[theorem]{Definition}
\newtheorem{example}[theorem]{Example}
\newtheorem{exercise}[theorem]{Exercise}
\newtheorem{lemma}[theorem]{Lemma}
\newtheorem{notation}[theorem]{Notation}
\newtheorem{problem}[theorem]{Problem}
\newtheorem{proposition}[theorem]{Proposition}
\newtheorem{remark}[theorem]{Remark}
\newtheorem{solution}[theorem]{Solution}
\newtheorem{summary}[theorem]{Summary}
\newenvironment{proof}[1][Proof]{\textbf{#1.} }{\ \rule{0.5em}{0.5em}}
\usepackage{indentfirst}

\newcommand{\Q}{\mathbb{Q}}
\newcommand{\R}{\mathbb{R}}
\newcommand{\C}{\mathbb{C}}
\newcommand{\Z}{\mathbb{Z}}
\renewcommand{\thesubsection}{\thesection.\alph{subsection}}
\setlength\parindent{24pt}

\begin{document}

\title{R/Stata Introduction}
\author{Alexandra Naumenko}
\date{}
\maketitle
\section{R vs Stata Introduction} 

\subsection{Advantages of Stata}

\begin{itemize}
\item user friendly
\item convenient canned packages for statistical analysis
\item quality control via StataCorp
\item most popular statistical programming software used in academia (for the social sciences)
\end{itemize}

\subsection{Advantages of R}

\begin{itemize}
\item free 
\item open source (code can easily be shared internationally)
\item advanced graphics/data visualization options
\item R is very marketable skill for data science oriented private sector jobs
\end{itemize}

\subsection{Objective}
We will perform the same exercise in both R and Stata so you can familiarize yourself with the syntax and get a sense of which one you would enjoy using more for assignments. 

\subsection{Resources}
\textbf{Stata}
\begin{itemize}
\item Stata Manual http://www.stata.com/manuals/u.pdf
\item the help command \\ example: \textit{help reg} will pull up more detailed information about executing a regression.
\end{itemize}

\textbf{R}
\begin{itemize}
\item \textit{?'command'} will pull up more detailed information about executing a regression. \textit{??`command'} performs a search. 
\item Heiss’ “Using R for Introductory Econometrics” (2016) available for a cheap price on
Amazon or as a free ebook on their webpage.
\end{itemize}


\pagebreak
\section{Example Routine in Stata}
Your version might not already have some packages installed so keep the following commands for installing useful packages in mind.
\begin{itemize}
\item ssc install estout, replace
\end{itemize}

\subsection {Importing and Viewing Data}
Start by opening a do file editor. The editor is useful for staying organized and having a script to look back to at a later time. The command window is mainly useful for experimenting with commands and using the help command. 
\begin{enumerate}
\item determine working directory \\
\textit{cd "insert file pathway here"}
\item I recommend always having \textit{clear} at the top of your do file so you can rerun the script smoothly
\item import the data\footnote{syntax varies slightly depending on the type of file e.g., .dta, .xls etc} \\ i\textit{import delimited mroz }
\item view the data to assess how effectively Stata read the data file \textit{browse}
\item assess summary stats \\ \textit{sum} 
\end{enumerate}
\subsection {Cleaning the Data}
\begin{enumerate}
\item Often you want to rename variables to keep syntax more compact or to clarify what the variable is. In our case, let's shorten hourstotalin1975 to hours.
\\
\textit{rename hourstotalin1975 hours}
\item Perhaps you want to drop missing values for wage.\footnote{you will learn in this class why this is often not a good idea}
\\
\textit{drop if wage == .}
\\item A useful addition to the dataset might be a dummy variable for whether the women works fulltime (i.e., 40 hours a week) \\ 
\textit{gen fulltime = hours>40*52}
\end{enumerate}

\subsection{Plotting data}
You suspect that education and wage should be highly correlated. You want to investigate this question with visuals. 
\\
\textit{reg wage educ\\
predict fitted\\
scatter wage educ || line fitted educ
}\\
\begin{figure}[h]\centering
\includegraphics[scale=.10]{Graph.png}
\end{figure}
\subsection{Analysis}
\begin{itemize}


\item We suspect that wage is not heavily distributed so we will look at a histogram \\ \textit{hist wage}
\begin{figure}[h]\centering
\includegraphics[scale=.10]{Graph1.png}
\end{figure}
\item Since the histogram validated our suspicion, we should log wage to correct for this. To simplify the analysis, we will restrict the sample to those who are employed (wage>0). All the log 0's would have resulting in missing values so the unemployed observations will drop out in the following regressions by default.\\ \textit{gen lwage = log(wage)}
\item Education clearly doesn't explain wage entirely on its own so we would benefit from adding a control. Let's add in experience. \\ \textit{reg lwage educ exper }
\end{itemize}

\subsection{Presenting results}
You are now happy with your analysis and you want to display your results in a table. Your audience will be interested in your significance levels so you want to make sure to include those!\\
\textit{\\
reg lwage educ exper\\
eststo clear\\
eststo: quietly reg wage educ\\
eststo: quietly reg wage educ exper\\
esttab}

\begin{figure}[h]\centering
\includegraphics[scale=.90]{Graph2.png}
\end{figure}


\includepdf[pages=-]{log_day1.pdf}

%%%%%%%%%%%%%%%%%%%%%%%%%%%%%%%%%%%%%%%%%%%%%%%%%%%%%%%%%%%%%%%%
%%%%%%%%%%%%%%%%%%%%%%%%%%%%%%%%%%%%%%%%%%%%%%%%%%%%%%%%%%%%%%%%

\section{Example Routine in R} 
Your version might not already have some packages used in this activity installed so keep the following commands for installing these packages in mind. 

\begin{lstlisting}[language=R]
install.packages('data.table')
install.packages('stargazer')
\end{lstlisting}

This is a useful site for learning about the various packages:\\ $https://cran.r-project.org/web/packages/available\_packages\_by\_name.html$

\subsection {Importing and Viewing Data}
Start by opening an R script. The editor is useful for staying organized and having a script to look back to at a later time. The console window is mainly useful for experimenting with commands and using the help command. 
\begin{enumerate}
\item determine working directory \\
\textit{setwd(insert your path here)}\footnote{it's annoying but R requires forward slashes in between folders}
\item import the data\footnote{syntax varies slightly depending on the type of file e.g., .dta, .xls etc} \\ i\textit{mroz <- read.csv("mroz.csv")}
\item view the data to assess how effectively R read the data file \textit{head(mroz)}
\item assess summary stats \\ \textit{summary(mroz)} 
\end{enumerate}

\subsection {Cleaning the Data}
\begin{enumerate}
\item Often you want to rename variables to keep syntax more compact or to clarify what the variable is. In our case, let's shorten hourstotalin1975 to hours.
\\
\textit{names(mroz)[names(mroz) == "hourstotalin1975"] <- "hours"}
\item Perhaps you want to drop missing values for wage.\footnote{you will learn in this class why this is often not a good idea}
\\
\textit{mroz2 <- na.omit(mroz)}
\\
\item A useful addition to the dataset might be a dummy variable for whether the women works fulltime (i.e., 40 hours a week) \\ 
\begin{lstlisting}[language=R]
{mroz2$fulltime <- mroz2$hours > 40*52}
\end{lstlisting}
\end{enumerate}

\subsection{Plotting data}
You suspect that education and wage should be highly correlated. You want to investigate this question with visuals. 
\\
\begin{lstlisting}[language=R]
reg_1 <- lm(wage ~ educ, data = mroz2) \\
plot(mroz2$educ, mroz2$wage) \\
abline(lm(wage ~ educ, data = mroz2))

\end{lstlisting}



\begin{figure}[h]\centering
\includegraphics[scale=.50]{Rplot.png}
\end{figure}
\subsection{Analysis}
\begin{itemize}


\item We suspect that wage is not heavily distributed so we will look at a histogram \\ 
\begin{lstlisting}{language=R}
hist(mroz2$wage)
\end{lstlisting}
\begin{figure}[h]\centering
\includegraphics[scale=.50]{Rplot2.png}
\end{figure}
\item Since the histogram validated our suspicion, we should log wage to correct for this. Unlike in Stata, we don't have to generate a new variable. We can just insert it into the regression command. To simplify the analysis, we will restrict the sample to those who are employed (wage>0). 

\item Education clearly doesn't explain wage entirely on its own so we would benefit from adding control. Let's add in experience. 
\item A benefit of R over Stata is that you can apply transformations to variables in the regression instead of generating them beforehand like you have to do in Stata\footnote{Stata will run using all the observations with missing values for lwage, but in R, you have to go out of your way to ensure your regression is only run on non-missing values to avoid an error.}
\end{itemize}

\begin{lstlisting}{language=R}
install.packages('data.table')
library(data.table)
mroz3 <- as.data.table(mroz2)
mroz3 <- mroz3[which(wage >0),]


reg_2 <- lm(log(wage) ~ educ, data = mroz3) 
reg_3 <- lm(log(wage) ~ educ + exper, data = mroz3)
\end{lstlisting} 
\subsection{Presenting results}
You are now happy with your analysis and you want to display your results in a table. Your audience will be interested in your significance levels so you want to make sure to include those!\\

\begin{lstlisting}{language=R}
install.packages('stargazer')
library(stargazer)
stargazer(reg_2,reg_3,type = "text")
\end{lstlisting}

\begin{table}[!htbp] \centering 
  \caption{} 
  \label{} 
\begin{tabular}{@{\extracolsep{5pt}}lcc} 
\\[-1.8ex]\hline 
\hline \\[-1.8ex] 
 & \multicolumn{2}{c}{\textit{Dependent variable:}} \\ 
\cline{2-3} 
\\[-1.8ex] & \multicolumn{2}{c}{log(wage)} \\ 
\\[-1.8ex] & (1) & (2)\\ 
\hline \\[-1.8ex] 
 educ & 0.109$^{***}$ & 0.110$^{***}$ \\ 
  & (0.014) & (0.014) \\ 
  & & \\ 
 exper &  & 0.016$^{***}$ \\ 
  &  & (0.004) \\ 
  & & \\ 
 Constant & $-$0.191 & $-$0.408$^{**}$ \\ 
  & (0.185) & (0.191) \\ 
  & & \\ 
\hline \\[-1.8ex] 
Observations & 425 & 425 \\ 
R$^{2}$ & 0.119 & 0.150 \\ 
Adjusted R$^{2}$ & 0.117 & 0.146 \\ 
Residual Std. Error & 0.680 (df = 423) & 0.669 (df = 422) \\ 
F Statistic & 57.099$^{***}$ (df = 1; 423) & 37.208$^{***}$ (df = 2; 422) \\ 
\hline 
\hline \\[-1.8ex] 
\textit{Note:}  & \multicolumn{2}{r}{$^{*}$p$<$0.1; $^{**}$p$<$0.05; $^{***}$p$<$0.01} \\ 
\end{tabular} 
\end{table} 

\pagebreak
.
\pagebreak 

\begin{lstlisting}{language=R}
setwd("C:/Users/ANaumenko/Dropbox/590")

mroz <- read.csv("mroz.csv")
head(mroz)
summary(mroz)
names(mroz)[names(mroz) == "hourstotalin1975"] <- "hours"

mroz2 <- na.omit(mroz)
mroz2$fulltime <- mroz2$hours > 40*52

reg_1 <- lm(wage ~ educ, data = mroz2)
plot(mroz2$educ, mroz2$wage)
abline(lm(wage ~ educ, data = mroz2))

hist(mroz2$wage)

library(data.table)
mroz3 <- as.data.table(mroz2)
mroz3 <- mroz3[which(wage >0),]


reg_2 <- lm(log(wage) ~ educ, data = mroz3) 
reg_3 <- lm(log(wage) ~ educ + exper, data = mroz3)

library(stargazer)
stargazer(reg_2,reg_3,type = "latex")

\end{lstlisting}




\end{document}